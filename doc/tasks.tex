\documentclass[a4paper,10pt]{article}
\usepackage[utf8x]{inputenc}

% Pacotes 
\usepackage[brazil]{babel} 
%\usepackage[latin1]{inputenc}
\usepackage{amsfonts}
\usepackage{amssymb} 
\usepackage{indentfirst} 	% indentacao do primeiro paragrafo
\usepackage{cite}  		% modo de citacao legal

%\usepackage{makeidx} 		% índice remissivo
\usepackage[nottoc]{tocbibind}  % acrescentamos a bibliografia/indice/conteudo no Table of Contents
\usepackage{setspace}

% By David
\usepackage[dvips]{graphicx} 	% ou usa [dvips]{graphicx} ou {pdfpages}
%\usepackage{pdfpages}		% permite o use de ``includepdf''
\usepackage{amsthm}
\usepackage{acronym} 
%\usepackage[portugues,ruled,vlined,linesnumbered]{algorithm2e/algorithm2e}
\usepackage{supertabular}
%\usepackage[margin=3cm,noheadfoot]{geometry}


%opening
\title{Planejamento do Projeto}
\author{}
\date{}

\begin{document}

\maketitle

\section{Atividades}

\begin{enumerate}
    \item[a.] {\bf Documentação}
	\begin{itemize}
	    \item {\bf Descrição}: Definição das especificações e caracteristicas do projeto.
	\end{itemize}

    \item[b.] {\bf Comunicação Wi-Fi com BlackWidow}
	\begin{itemize}
	    \item {\bf Descrição}: Teste de comunicação do Arduino BlackWidow utilizando Wi-fi.
	    \item {\bf Tarefas}:
		\begin{itemize}
			\item Definição da arquitetura a ser utilizada;
			\item Definição de um protocolo de comunicação;
			\item Implementação do código responsavél pela comunicação, tanto no servidor quanto no cliente;
			\item Definição do procedimento para repetição do teste;
			\item Documentação dos resultados obtidos.
		\end{itemize}
	\end{itemize}

    \item[c.] {\bf Controle de uma matriz de LEDs RGB utilizando Rainbowduino}
	\begin{itemize}
	    \item {\bf Descrição}: Teste de controle de uma matriz de LEDs utilizando um Rainbowduino.
	    \item {\bf Tarefas}:
		\begin{itemize}
		    \item Definição da arquitetura a ser utilizadas;
		    \item Implementação do código responsavél pelo controle dos LEDs;
		    \item Definição do procedimento para repetição do teste;
		    \item Documentação dos resultados obtidos.
		\end{itemize}
	\end{itemize}

    \item[d.] {\bf Controle de uma pequena quantidade de LEDs RGB utilizando Rainbowduino}
	\begin{itemize}
	    \item {\bf Descrição}: Teste de controle de um conjunto de 6 (seis) de LEDs RGB utilizando um Rainbowduino.
	    \item {\bf Tarefas}:
		\begin{itemize}
		    \item Definição da arquitetura a ser utilizadas;
		    \item Implementação do código responsavél pelo controle dos LEDs;
		    \item Implementação do circuito responsavél pela conexão com os LEDs;		    
		    \item Definição do procedimento para repetição do teste;
		    \item Documentação dos resultados obtidos.
		\end{itemize}
	\end{itemize}    
    
    \item[e.] {\bf Cascateamento de Rainbowduinos}
	\begin{itemize}
	    \item {\bf Descrição}: Teste de utilização de um cascateamento de dois (2) Rainbowduinos, onde cada um deles controla um conjunto de LEDs RGB.
	    \item {\bf Tarefas}:
		\begin{itemize}
		    \item Definição da arquitetura a ser utilizadas;
		    \item Implementação do código responsavél pelo controle dos LEDs;
		    \item Implementação do circuito responsavél pela conexão com os LEDs;		    
		    \item Definição do procedimento para repetição do teste;
		    \item Documentação dos resultados obtidos.
		\end{itemize}
	\end{itemize}
    
    
    \item[f.] {\bf Definição dos protocolos de comunicação}
	\begin{itemize}
	    \item {\bf Descrição}: Definição dos protocolos de utilização no projeto \ldots
	    \item {\bf Tarefas}:
		\begin{itemize}
		    \item \ldots
		    \item Documentação dos protocolos.
		\end{itemize}
	\end{itemize}

\end{enumerate}

\section{Cronograma}

\begin{tabular}{|c|c|c|c|c|c|c|c|}
\hline
              & {\bf a.}     & {\bf b.}     & {\bf c.}     & {\bf d.}     & {\bf e.}     & {\bf f.}     & {\bf g.}     \\ \hline 
{\bf Semana}  & Todos        & Douglas      & David        & David        & David        &              &              \\ \hline \hline
15/08 - 21/08 & $\checkmark$ & $\checkmark$ & $\checkmark$ &              &              &              &              \\ \hline
22/08 - 28/08 & $\checkmark$ & $\checkmark$ & $\checkmark$ & $\checkmark$ &              &              &              \\ \hline
29/08 - 05/09 & $\checkmark$ &              &              & $\checkmark$ & $\checkmark$ &              &              \\ \hline
05/09 - 11/09 & $\checkmark$ &              &              &              & $\checkmark$ &              &              \\ \hline
12/09 - 18/09 & $\checkmark$ &              &              &              &              &              &              \\ \hline
19/09 - 25/09 & $\checkmark$ &              &              &              &              &              &              \\ \hline
26/09 - 02/10 & $\checkmark$ &              &              &              &              &              &              \\ \hline
03/10 - 09/10 & $\checkmark$ &              &              &              &              &              &              \\ \hline
10/10 - 16/10 & $\checkmark$ &              &              &              &              &              &              \\ \hline
17/10 - 23/10 & $\checkmark$ &              &              &              &              &              &              \\ \hline
24/10 - 30/10 & $\checkmark$ &              &              &              &              &              &              \\ \hline
31/10 - 06/11 & $\checkmark$ &              &              &              &              &              &              \\ \hline
07/11 - 13/11 & $\checkmark$ &              &              &              &              &              &              \\ \hline
14/11 - 20/11 & $\checkmark$ &              &              &              &              &              &              \\ \hline
21/11 - 27/11 & $\checkmark$ &              &              &              &              &              &              \\ \hline
28/11 - 04/12 & $\checkmark$ &              &              &              &              &              &              \\ \hline
\end{tabular}

\ldots

\end{document}
